\documentclass[11pt,onecolumn,superscriptaddress,notitlepage]{article}

\usepackage[total={6.5in,9in}, top=1.0in, includefoot]{geometry}
\usepackage{epsfig}
\usepackage{subfigure}
\usepackage{placeins}
\usepackage{amsmath}
\usepackage[usenames,dvipsnames,svgnames,table]{xcolor}
\usepackage{amssymb}
\usepackage{setspace}
\usepackage{graphicx} % Include figure files
\usepackage{times}
\usepackage{amsthm}
\usepackage{hyperref}
\usepackage[affil-it]{authblk} 
\hypersetup{bookmarks=true, unicode=false, pdftoolbar=true, pdfmenubar=true, pdffitwindow=false, pdfstartview={FitH}, pdfcreator={Daniel Larremore}, pdfproducer={Daniel Larremore}, pdfkeywords={} {} {}, pdfnewwindow=true, colorlinks=true, linkcolor=red, citecolor=Green, filecolor=magenta, urlcolor=cyan,}

\usepackage{enumitem}

\newcommand{\dx}[0]{\displaystyle\frac{d}{dx}}

\usepackage{parskip}

\date{}
\begin{document}

%%%%%%%%%% Authors
\author{CSCI 2897 - Calculating Biological Quantities - Larremore - Fall 2022}
%%%%%%%%%% Title
\title{Assignment 1}
\maketitle
%%%%%%%%%% Content

    %%%    
    %%%   
  %%%%%%%
   %%%%%
    %%%
     %
{\bf Notes:} Remember to (1) familiarize yourself with the collaboration policies posted on the Syllabus, and (2) turn in your homework to Canvas as a {\bf single PDF}. Hand-writing some or most of your solutions is fine, but be sure to scan and PDF everything into a single document. Unsure how? Ask on Slack! 

\section*{Squats}

{\bf Calculate these derivatives.}

\begin{enumerate}
	\item $\dx x^3 = $
	\item $\dx x^{-3} = $
	\item $\dx e^{\alpha x} = $
	\item $\dx e^{\pi x^{-2}} = $
	\item $\dx \ln{2x} = $
\end{enumerate}

\section*{Situps}

{\bf Find solutions to each of these differential equations.}\footnote{Hint: ask yourself, ``What function, if I were to take its derivative, would satisfy this equation?''}

\begin{enumerate}[resume]
	\item $\displaystyle\frac{dy(t)}{dt} = 0$
	\item $\displaystyle\frac{dy(t)}{dt} = t$
	\item $\displaystyle\frac{dy(t)}{dt} = y(t)$
\end{enumerate}

\clearpage
\section*{Modeling in the News}
\begin{enumerate}[resume]
\item Find {\bf two} stories in the recent news that spark your curiosity about modeling, one related to biology in some form, and another unrelated to biology. For each, please
\begin{itemize}
	\item Provide a link to the story, as well as the date and title of the story. 
	\item Write a paragraph describing {\it as a narrative} a dynamical process occurring in the story. 
	\item Pose a relevant question about that dynamical process or system.
	\item Identify the important variables; and identify the important parameters.
	\item Produce a flow diagram or a life cycle diagram of the dynamics using a graphics software\footnote{Keynote or Powerpoint are good bets} that would help you to translate the process or system from narrative steps (qualitative) into a quantitative model with variables and parameters included.
\end{itemize}
\end{enumerate}

\vspace{0.3in}
\section*{Minors and Majors}
\begin{enumerate}[resume]
\item Each year, the Computational Biology Minor has $N$ new enrollees who start as freshmen. These freshmen are split with p\% majoring in Computation ($C$) and 100(1-p)\% majoring in Biology ($B$). At the end of freshman year, sophomore year, and junior year, a fraction $f_{C\to B}$ of Computation students change to Biology, while a fraction $F_{B\to C}$ Biology students change to Computation. Also at the end of each year, a fraction $f_X$ of the students drop the CB Minor entirely. The remaining students keep their existing major and show up in the fall in the next grade; Seniors graduate and leave.

Draw a {\bf flow diagram} that tracks the numbers of students in Computation across the four years ($C_f$, $C_s$, $C_j$, $C_r$) and the numbers of students in Biology across the four years ($B_f$, $B_s$, $B_j$, $B_r$). Include parameters in your diagram. State any fundamental requirements on the parameters that you can think of. 
\end{enumerate}

\vspace{0.5in}
\section*{Extra Credit}
\begin{enumerate}[resume]
\item[E.C.] As noted in class, we can use {\it Forward Euler} to numerically solve the differential equation$$\frac{dn(t)}{dt}=\sqrt{n(t)},\quad n(0)=1$$
by determining our current ``slope'', and then taking a small step ($\Delta t$) in that direction to update the value of $n$. In this way, we can step along the path of the solution, and solve a differential equation by transforming it into a recursion. 

For this extra credit, write some code in Python and produce a single plot that shows three solutions: (a) $\Delta t = 2$, red, (b) $\Delta t = 1$, blue, and (c) $\Delta t = 0.01$, black.  Your plot should have a horizontal axis from $t=0$ to $t=10$. Please also attach your source code along with your plot --- a screenshot of your code is fine. 
\end{enumerate}


     %
    %%%
   %%%%%
  %%%%%%%
    %%%
    %%%

\end{document}