\documentclass[11pt,onecolumn,superscriptaddress,notitlepage]{article}

\usepackage[total={6.5in,9in}, top=1.0in, includefoot]{geometry}
\usepackage{epsfig}
\usepackage{subfigure}
\usepackage{placeins}
\usepackage{amsmath}
\usepackage[usenames,dvipsnames,svgnames,table]{xcolor}
\usepackage{amssymb}
\usepackage{setspace}
\usepackage{graphicx} % Include figure files
\usepackage{times}
\usepackage{amsthm}
\usepackage{hyperref}
\usepackage[affil-it]{authblk} 
\hypersetup{bookmarks=true, unicode=false, pdftoolbar=true, pdfmenubar=true, pdffitwindow=false, pdfstartview={FitH}, pdfcreator={Daniel Larremore}, pdfproducer={Daniel Larremore}, pdfkeywords={} {} {}, pdfnewwindow=true, colorlinks=true, linkcolor=red, citecolor=Green, filecolor=magenta, urlcolor=cyan,}

\usepackage{enumitem}

\newcommand{\dx}[0]{\displaystyle\frac{d}{dx}}
\newcommand{\dy}[0]{\displaystyle\frac{dy}{dt}}

\usepackage{parskip}

\date{}
\begin{document}

%%%%%%%%%% Authors
\author{CSCI 2897 - Calculating Biological Quantities - Larremore - Fall 2022}
%%%%%%%%%% Title
\title{Assignment 2}
\maketitle
%%%%%%%%%% Content

    %%%    
    %%%   
  %%%%%%%
   %%%%%
    %%%
     %
{\bf Notes:} Remember to (1) familiarize yourself with the collaboration policies posted on the Syllabus, and (2) turn in your homework to Canvas as a {\bf single PDF}. Hand-writing some or most of your solutions is fine, but be sure to scan and PDF everything into a single document. Unsure how? Ask on Slack! 

\section*{Bicep curls}

{\bf Calculate these derivatives.}

\begin{enumerate}[itemsep=35pt]
	\item $\dx \cos(2x) = $
	\item $\dx \cos(x^2) = $
	\item $\dx xe^x = $
	\item $\dx \ln(x^2) = $
	\item[EC A.] $\dx \left [ \sin(x)\cos(x) \right ] = $
\end{enumerate}

\clearpage
\section*{Tricep extensions}

{\bf Calculate these indefinite integrals. Don't forget your constant!} 

\begin{enumerate}[resume,itemsep=90pt]
	\item $\displaystyle\int x^2\ dx=$
	\item $\displaystyle\int x^{-2}\ dx = $
	\item $\displaystyle\int e^{2 \pi x}\ dx = $
	\item $\displaystyle\int \sin{x}\ dx = $
	\item[EC B.] $\displaystyle\int \sin{x}\ \csc{x}\ dx = $
\end{enumerate}

\clearpage
\section*{Planks} 

{\bf For each family of solutions below, (i) use the {\it initial condition} to solve for the unknown constant $\alpha$, and then (ii) write the solution with the solved-for constant plugged in and simplified.}

\begin{enumerate}[resume,itemsep=90pt]
	\item $y(t) = \alpha e^{3t}, \quad y(0) = 10 $
	\item $y(t) = \alpha e^{t/2}, \quad y(6) = e $
	\item $\displaystyle n(t) = \frac{K}{1+\alpha Ke^{-r t}}, \quad n(0) = 1$
	\item $\displaystyle y(x) = \frac{1}{x^2 + \alpha}, \quad y(2) = \frac{1}{3}$  
	\item[EC C.] $\displaystyle n(t) = \frac{K}{1+ \alpha Ke^{-r t}}, \quad n(0) = K$
\end{enumerate}


\clearpage
\section*{Separation of Variables}

{\bf Classify each equation as separable or not separable.}  Then, for separable equations, separate the variables {\it but do not integrate.}
\begin{enumerate}[resume,itemsep=90pt]
	\item $t \dy = 4y$
	\item $t \dy = 4+t$
	\item $y \ln t \dy = \left(\frac{y+1}{t}\right)^2$
	\item $e^t + e^y = \dy$
	\item[EC D.] $e^{t+y} = \dot{y}$
\end{enumerate}


     %
    %%%
   %%%%%
  %%%%%%%
    %%%
    %%%

\end{document}